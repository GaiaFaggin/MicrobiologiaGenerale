\chapter{Meccanismi di trasporto cellulare}
La membrana media un'ampia varietà di funzioni. Una delle funzioni pi\`u importanti \`e il controllo di ci\`o  che entra ed esce dalla cellula.La membrana durante questo processo è molto selettiva: permette ad alcune membrane di oltrepassarla, mentre lo nega ad altre. Esistono due tipi diversi di diffusione: uno che usufruisce di energia e un altro che non ne fa uso.
\section{Processi passivi}
In questo tipo di processi non c'\`e utilizzo di energia. Le molecole passsano da un ambiente dove la loro concentrazione \`e maggiore a una dove la loro concentrazione \`e minore.
\subsection{Diffusione passiva}
Le molecole si muovono dalla regione a pi\'u alta concetrazione a quella a pi\'u bassa per agitazione termica. $H_2$O, $O_2$ e $CO_2$ sono molecole piccole e apolari e per questo motivo passano liberamente attraverso il doppio strato lipidico. 
\subsection{Diffusione facilitata}
\`E simile alla diffusione passiva. La direzione del movimento \`e sempre da alta a bassa concentrazione e il valore del gradiente di concentrazione incide sul tasso di assorbimento. Vengono utilizzate delle molecole trasportatrici (carriere o permeasi) ed \`e necessario un minore gradiente di concentrazione per un significativo assorbimento delle molecole. Trasporta glicerolo, zuccheri e amminoacidi.
\subsection{Passaggio specifico}
\subsection{Osmosi}
L'osmosi \`e la diffusione di acqua attraverso una membrana selettivamente permeabile, dal compartimento con la minor concentrazione di soluti verso quello con la maggior concentrazione. \'E un fenomeno fondamentale di regolazione nella cellula: essa tende a rendere la concentrazione di soluti presenti all'interno della cellula pari a quella presente nell'ambiente extracellulare. In base alle soluzioni in cui la cellula \`e immersa ne esistono tre tipi:
\begin{itemize}
    \item isotonica: nessun movimento netto di acqua;
    \item ipertonica: uscita di acqua causando la diminuzione del volume cellulare;
    \item ipotonica: ingresso di acqua portandoalla lisi delle cellule prive di parete.
\end{itemize}
\section{Trasporto attivo}
Il trasporto attivo richiede energia per movimentare sostanze contro il proprio gradiente di concentrazione. L'energia viene fornita dall'idrolisi dell'ATP o dalla forza proton-motrice.
I trasportatori che attraversano la membrana sono costituiti da 12\alphaA eliche che si aggregano in modo da formare un canale attraverso la membrana. 
I tre principali tipi di trasporto attivo sono: 
\begin{itemize}
        \item Uniporto: permette il passaggio di una molecola specifica solamente in una direzione, anche se questo significa andare contro gradiente. 
        \item Antiporto: è il trasporto contemporaneo di due specie ioniche o di altri saluti che si muovono in senso opposto. Una delle due sostanze viene fatta lasciata passare secondo gradiente.Questo genera l'energia entropica necessaria per far passare l'altra sostanza contro gradiente.
        \item Simporto: funziona analogamente all'antiporto, ma ora le due sostanze attraversano la membrana nella stessa direzione. (Esempio. La Lac permeasi richiede energia per importare il lattosio nella cellula. Man mano che trasposrta il lattosio, l'energia della forza proton-motrice si riduce a causa del trasporto simultaneo di protoni all'interno della cellula.)
\end{itemize}
\section{Traslocazione di gruppo}
La traslocazione di gruppo differisce dal trasporto semplice per due motivi: 
\begin{enumerate}
    \item durante il processo la molecola viene modificata chimicamente;
    \item il processo viene azionato da un composto organico ricco di energia piuttosto che dalla forza proton-motrice.
\end{enumerate}
Uno dei processi di traslocazione di gruppo più studiati \`e il trsaporto degli zuccheri glucosio, maltosio e fruttosio in E. coli. Queste sostanza vengono fosforilate durante il trasporto dal sistema delle fosfotransfersi. Consiste di una famiglia di proteine che lavorano in stretta intesa. Per la cattura del glucosio, il sistema necessita di cinque proteine: Enz I, Enz IIa, Enz IIb, Enz IIc e HPr. Il trasferimento a cascata del fosfato avviene a partire dal fosfoenolpiruvato fino all'EnzimaIIc. Quest'ultimo trasporta e fosforila lo zucchero. Le proteine HPr e Enz I non sono specifiche e sono coinvolte nel trasporto di tutti gli zuccheri. I componenti di Enz II sono specifici per uno zucchero particolare.
\subsection{Proteine di legame periplasmatiche e sistema ABC}
I batteri gram-negativi possiedono uno spazio chiamato peripalsma tra la maembrana citoplasmatica e la membrana esterna. Questo spazio \`e occupato per la maggiorparte da proteine di legame periplasmatiche. 
Nei procarioti sono stati identificati pi\`u di 200 diversi sistemi di traporto ABC. Le proteine di legame periplasmatico mostrano un'elevata affinit\`a per il loro substrato. Una volta che che il subrato viene sequestrato, forma un complesso che interagisce con la proteina integrale di membrana. Il processo \`e altamente specifico per diversi tipi di substrato, come zuccheri, amminoacidi, solfati, fosfati, metalli. Le proteine periplasmatiche possono legare i substrati anche quando sono a concentrazioni molto basse, per esempio a 1 micromolare.
Anche se i batteri gram-positivi sono privi di periplasma hanno i sistemi di trasporto ABC. In questi batteri le proteine sono ancorate alla superficie esterna della membrana citoplasmatica. 
Nei batteri gram-negativi questo sistema viene utilizzato per il trasporto del ferro. Dato che il metallo deve attraversare due membrane. Esistono due tipi diversi di trasporto in ognuna di esse: 
\begin{itemize}
    \item il ferro si lega ad una molecola trasportatrice e questo complesso attraversa la membrana esterna grazie alla forza protonmotrice degli ioni H\ap{+};
    \item poi riesce a oltrepassare quella interna con l'auito di un complesso ABC. 
\end{itemize}
Una volta all'interno della cellula il trasportatore si stacca dal ferro e lo rilascia. 
\section{Traslocazione}
La traslocazione \`e il trasferimento di una proteina da un compartimento a un altro. Molte proteine necessitano di essere trasportate fuori dalla cellula o di essere inserite in modo specifico nella membrana. Nei procarioti l'esportazione delle proteine avviene attraverso l'attivit\`a di proteine dette traslocasi. Una sequenza N-terminale viene  riconosciuta da delle proteine specifiche ancora quando la traduzione della proteina deve essere completata. La sequenza segnale pu\`o essere variabile ed \`e composta da 15-30 aminoacidi: porzione idrofila basica + regione idrofoba + porzione polare.
\section{Secrezione}
Durante la secrezione la proteia viene secreta completamente nell'ambiente esterno, senza che abbia alcun legame con la cellula da cui proveniva. 
\\Ancora prima che la proteina venga formata completamente, la proteina \textit{SecA} roconosce la sequenza segnale. Solitamente la sequenza segnale \`e formata da alcuni residui di aminoacidi carichi positivamente nell'N-terminale, seguiti da aminoacidi idrofobici e residui polari. Esitono molti tipi di questa proteina e il suo attacco alla sequenza sengnale non provoca alcun ritardo nella traduzione della proteina. Dopo il riconoscimento, \textit{SecA} trasporta la proteina attraverso un canale idrofobico formato da altre proteine \textit{Sec}.
\\Invece, \textit{SecB} ha un ruolo importante nel rendere pi\`u distesa la struttura della proteina, tale da consentirne il passaggio attraverso il canale e impedendone il suo ripiegamento prima della secrezione. Svolge questo ruolo utilizzando l'energia rilasciata dall'idrolisi di ATP.
\\Poi \textit{SecA} riconosce il recettore specifico situato nella membrana plasmatica con l'aiuto una decina di altre proteine. La proteina \textit{Lep} \`e una peptidasi del segnale, la quale taglia la sequenza segnale all'interno della cellula. Questa viene poi degradata e fli aminoacidi da cui era composta vengono riutilizzati per la sintesi di nuove proteine.
\section{Esportazione}
Anche in questo caso la proteina non ancora formata espone un segnale di ancoraggio nel suo N-terminale che le permette di essere riconosciuta come una proteina che deve essere esportata. 
La \textit{SRP} (\textit{Signal Recognisation Particel}) attua questo riconoscimento e non permette alla proteina di ripiegarsi prima di essere esporatata. \`E formata da una proteina (\textit{Ffh}) e una molecola di RNA 4.5S (\textit{ffs}). Dopo essere stata riconìosciuta dalla SRP, trasporta la proteina ancora non completamente formata ad un recettore specifico. 
\\Dopo la proteina viene inserita e rimane attaccata alla membrana, l'isnerimento avviene in due modi:
\begin{itemize}
    \item aiutato dal complesso \textit{SecY} e \textit{SecE}; 
    \item senza l'utilizzo del complesso sopra citato.
\end{itemize}
Molte volte la sequenza segnale non viene tagliata e si forma una proteina transmembrana. Il compleso nel totale \`e continuo e contraduzionale, cio\`e in contemporanea con la sintesi proteica. Non necessita di molecole che aiutino il ripiegamento appropriato  della proteina in formazione. Anche nel momento in cui la proteina deve attraversare due membrane, questa porta un'unica sequenza segnale. 
\section{Sistemi di secrezione}
I sistemi di secrezione presentno una grande variabilit\`a e permettono di attraversare la membrana esterna dei gram-negativi. Possono essere di due tipi: Sec-dipendenti o Sec-indipendenti.
\begin{itemize}
    \item \textbf{Tipo I} $\xrightarrow{}$ 
    \item \textbf{Tipo II} $\xrightarrow{}$ 
    \item \textbf{Tipo III} $\xrightarrow{}$ 
    \item \textbf{Tipo IV} $\xrightarrow{}$
    \item \textbf{Tipo V} $\xrightarrow{}$ 
\end{itemize}
