\chapter{Metabolismo dei microrganismi}
Il metabolismo \`e un insieme di reazioni biocimiche controllate. Alcuni degli elementi fondamentali sono:
\begin{itemize}
    \item nutrienti: elementi chimici essenziali; 
    \item energia: ricavata dalla luce o dalla degradazione dei nutrienti;
    \item enzimi: catabolizzano e anabolizzano i nutrienti;
    \item macromolecole: assemblaggio e polimerizzazione partendo da monomeri; 
    \item struttura cellulare: assemblaggio di pi\`u macromolecole. 
\end{itemize}
Il metabolismo viene diviso in due principali tipi di reazioni: cataboliche e anaboliche. 
Il catabolismo \`e la parte del cataolismo che libera energia, grazie alla scomposizione di molecole organiche. Una parte dell'energia viene conservata sotto forma di legami nell'ATP, mentre altra viene dispersa come calore. \`E una via esoergonica.
L'anabolismo utilizza l'energia che viene liberata dal catabolismo per creare molecole pi\`u grandi. Anche in questo tipo di processi viene persa dell'energia sotto forma di calore. \`E una via endoergonica.
Una via metabolica non \`e costituita da una sola reazione, ma da una serie complessa di reazioni. Durante una reazione chimica l'energia libera \`e definita come energia rialsciata diponibile per compiere un lavoro utile. 
Se ${\Delta}$G\ap{0'} \`e negativo significa che essa proceder\`a con liberazione di energia libera.
Se ${\Delta}$G\ap{0'} \`e positivo la reazione per aver luogo richiede energia.
\section{Catalisi ed enzimi}
Il calcolo dell'energia libera ci dice solo se in una certa reazione l'energia sia liberata o richiesta, ma non ci viene detto nulla sulla velocit\`a di reazione.
Consideriamo la reazione di formazione dell'acqua con ${\Delta}$G\ap{0'} = -237kJ. Le reazione \`e energeticamente favorevole, ma se noi mescoliamo ossigeno e idrogeno all'interno di una bottiglia l'acqua non si forma, perch\`e \`e necessario rompere prima i legami dei reagenti. Per rompere questi legami \`e necessaria l'energia di attivazione. Gli enzimi sono dei catalizzatori che diminuiscono l'energia di attivazione e quindi aumentano la velocit\`a della reazione.
\subsection{Enzimi}
Gli enzimi sono i catalizzatori biologici. Sono proteine, o raramente RNA, altamente specifici per la reazione da essi catalizzata. Quindi ciascun enzima catalizza un solo tipo di reazione oppure una classe di reazioni strettamente affini. Questa sua specificit\`a dipende dalla sua struttura tridimensionale. In una reazione catalizzata da un enzima (E), questo si combina con il substrato (S) formando un complesso (E-S). Mentre questa reazione procede viene rilasciato il prodotto(P) e l'enzima torna allo stato originale.
\\
\begin{center}E+S $\leftrightarrow$ E-S $\leftrightarrow$ E+P\end{center}
\\
Normalmente l'enzima \`e molto pi\`u grande del substrato e il la porzione a cui si lega il substrato \`e chiamato sito attivo.
\\Molti enzimi contengono delle piccole molecole non proteiche che partecipano alla funzione catalitica. Molti enzimi sono composti da pi\`u elementi organici e inorganici. L'apoezima, la porzione proteica degli enzimi, \`e attiva solo se associata a cofattori. I cofattori possono essere di due tipi: molecole inorganiche, normalmente ioni metalicci quali il ferro, il magnesio, lo zinco o il rame; oppure molecole organiche chiamate coenzimi. Questi ultimi sono o contengono vitamine. Queste sono delle molecole indispensabili al metabolismo e che molti organismi non sono in grado di sintetizzare, per questo motivo vanno assunte con la dieta. La forma completa e attiva dell'enzima \`e detta oloenzima.
Esistono diversi tipi di enzimi e la maggiorparte dei nomi degli enzimi contiene il suffisso "-asi" e spesso fa riferimento al tipo di substrato e di reazione biochimica mediata:
\begin{itemize}
    \item idrolasi $\xrightarrow{}$ catalizzano la rottura di un legame chimico con l'intevento di una molecola d'acqua (catabolismo);
    \item isomerasi $\xrightarrow{}$ catalizzano l'interconversione tra due isomeri (n\`e catabolismo, n\`e anabolismo);
    \item ligasi e polimerasi $\xrightarrow{}$ assemblano molecolaìe della stessa natura chimica (anabolismo);
    \item liasi $\xrightarrow{}$ catalizzano la rottura di diversi legami chimici attraverso processi differenti dall'idrolisi e dalla ossidazione (catabolismo);
    \item ossidoriduttasi $\xrightarrow{}$ catalizzano il trasferimento di elettroni da una molecola (donatrice di elettroni) ad un'altra (accettore di elettroni) (catabolismo o anabolismo);
    \item trasferasi $\xrightarrow{}$ spostano gruppi funzionali (amino, fosfato, acetile, etc.) da una molecola all'altra (anabolismo).
\end{itemize}
\subsection{Parametri esterni che alterano l'azione degli enzimi}
Temperatura $\xrightarrow{}$ L'alzarsi della temperatura tende ad incrementare la velocit\`a delle reazioni biochimiche. Tuttavia, le reazioni enzimatiche hanno un range di temperatura in cui possono svolgersi. Sopra una determinata temperatura i legami non-covalenti dell'enzima si rompono ed esso si denatura, portando alla perdita della struttura tridimensionale e della funzionalit\`a. La denaturazione pu\`o essere permanente p reversibile a seconda degli enzimi.
\\pH $\xrightarrow{}$ valori estremi di pH portano alla denaturazione quando gli ioni rilasciati da acidi o basi interferiscono con i legami idrogeno che assicurano la struttura tridimensionale dell'enzima.
\\Concentrazione del substrato o dell'enzima $\xrightarrow{}$ l'attivit\`a enzimatica aumenta in funzione della concentrazione di substrato fino a raggiungere il punto di saturazione quando tutti i siti attivi sono legati al substrato.
\subsection{Inibitori}
Gli inbitori hanno il compito di regolare l'attvit\`a enzimatica. Ne esistono di vario tipo:
\begin{itemize}
    \item Competitivi $\xrightarrow{}$ sono in grado di legare il sito attivo e presentano una forma e struttura chimica simile a quella del substrato. Competono con il substrato per il sito attivo dell'enzima. In linea generale, la loro inibizione \`e reveribile e pu\`o essere superata con l'aumento della concentrazione del substrato. Un esempio \`e la sulfanilamide che presenta una forte affinit\`a per il sito attivo dell'enizma che catalizza la conversione del PABA in acido folico, un precursore dei nucleotidi fondamentale per la sintesi del DNA.
    \item Non-competitivi $\xrightarrow{}$ Non legano il sito attivo ma un'altra regione chimata sito allosterico, portando ad un cambiamento conformazionale del sito attivo. Esistono due forme: inibitorie ed eccitatorie; alcuni le possegono entrambi.
    \item Feedback negativo
    $\xrightarrow{}$ regola la quantit\`a di una certa sostanza in base alla sua concentrazione. Il prodotto finale della via metabolica \`e un inibitore allesterico di un enzima che interviene pi\`u a monte nel pathway. Dato che il prodotto di ogni reazione \`e anche il substrato della successiva, l'intero pathway viene disattivato quando il prodotto finale \`e presente in concentrazione sufficiente. In una via ramificata ciascuno dei tre prodotti finali inibisce una delle tre enzimi sintetasi. Solo quando tutti i tre prodotti finali sono presenti in concentrazione deguata la sintesi del loro precursore, DHAP, viene interrotta. 
\end{itemize}
\section{Redox}
Molte reazioni metaboliche prevedono il trasferimento di una molecola (e\ap{-} donor) ad un'altra (e\ap{-} acceptor). L'acceptor viene ridotto perch\`e il guadagno di elettroni riduce la sua carica elettrica totale. Le moecole che predono elettroni si ossidano perch\`e spesso cedono ossigeno. Un esempio di reazione di ossido-riduzione \`e quella per la formazione dell'acqua: per ogni ossidazione occorre che avvenga una conseguente riduzione.
\\Potenziale di riduzione standard ($E_0$) $\xrightarrow{}$ \`e la tendenza di una sostanza di ossidarsi e ridursi. \`E espresso in Volt e prende come riferimento una sostanza standard $H_2$. Se $E_0$ \`e pi\`u negativo \`è un miglior donatore di elettroni; mentre se \`e pi\`u positivo \`e un miglior accettatore di elettroni.
\\La torre degli elettroni rappresenta il campo dei potenziali di riduzione possibili per le coppie redox in natura da quelle con $E\ap{I}_0$ pi\`u negativo in cima alla torre  a quelle con il valore di $E\ap{I}_0$ pi\`u positivo alla sua base. La sostanza ridotta nella coppia redox posta in cima alla torre redox ha la massima tendenza a donare elettroni, mentre la sostanza ossidata nella coppia redox sul fondo della torre redox ha la massima tendenza ad accettare elettroni. 
\section{Trasportatori di elettroni e ciclo NAD/NADH}
Le reazioni redox sono mediate da piccole molecole. Un intermedio redox molto comune \`e il NAD\ap{+} (nicotinammide adenina dinucleotide) e la sua forma ridotta NADH. NAD\ap{+} \`e un coenzima e trasportatore di elettroni, NADH \`e la sua forma ridotta. NADP\ap{+} e NADPH hanno solamente un gruppo fosfato in pi\`u. Entrambe le forme ridotte sono buoni donatori di elettroni ($E\ap{'}_0$ = -0,32). NAD\ap{+}/NADH \`e coinvolta nelle reazioni cataboliche che generno energia, mentre NADP\ap{+}/NADPH \`e coinvolta nelle reazioni biosintetiche (anaboliche).
\\NAD\ap{+} e NADH sono coenzimi e per questo motivo facilitano il lavoro di un enzima senza essere consumati. Alcuni di questi si legano al NAD\ap{+} e se incontrano un substrato adatto, legano anche quello in prossimit\`a del coenzima non ridotto. A questo punto in NAD prende due elettroni e un H\ap{+} e diventa NADH e cambia anche la conformazione del substrato. Se invece NADH si lega ad un enzima col suo substrato specifico, accade il contrario: viene donato un H\ap{+} e si trasforma in NAD\ap{+}.
\section{ATP}
L'adenosina trifosfato (ATP) \`e il pi\`u importante composto fosforilato; \`e costituito dal ribonucleoside adenosina a cui sono legati in serie tre molecole di fosfato. Viene generato durante le reazioni esoergoniche e consumato nelle reazioni endoergoniche. 
\\Nel catabolismo l'energia rilasciata dalla degradazione dei nutrienti viene stoccata nei legami ad alta energia tra i gruppi fosfato della molecola di ATP. Si forma dalla fosforilazione dell'ADP.
\\Le cellule fosforilano l'ADP per formare ATP in tre modi:
\begin{itemize}
    \item Fosforilazione a livello di substrato $\xrightarrow{}$ prevede il trasferimento del fosfato da una molecola organica all'ADP per formare ATP;
    \item Fosforilazione ossidativa $\xrightarrow{}$ l'energia derivata da reazioni redox della respirazione cellulare viene utilizzata per aggiungere fosfato inorganico ($P_i$) all'ADP; \item Fotofosforilazione $\xrightarrow{}$ l'energia luminosa viene utilizzata per fosforilare l'ADP con fosfato inorganico.
\end{itemize}
\section {Catabolismo dei carboidrati}
Il glucosio ed altri zuccheri vengono catabolizzati dai microrganismi tramite due processi: 
\begin{itemize}
    \item la respirazione cellulare $\xrightarrow{}$ consiste nella completa demolizione del glucosio per formare anidride carbonica ed acqua;
    \item fermentazione $\xrightarrow{}$ produce molecole organiche di scarto.
\end{itemize}
Entrambi i processi inziano con la glicolisi nel quale ogni molecola viene catabolizzata in due molecole di piruvato con la produzione netta di 2 molecole di ATP. La respirazione cellulare, poi, prosegue con il ciclo di Krebs e la catena di trasporto elettronico con sostanziale produzione di ATP; mentre la fermentazione converte l'acido piruvico in altre molecole organiche senza produzione di ATP. Nello stesso organismo \`e possibile utilizzare sia la respirazione cellulare che la fermentazione.
\section{Glicolisi}
La glicolisi, anche detta Via di Embden-Meyerhof-Parnas, \`e il primo passo per la metabolizzazione del glucosio. Questo processo scinde il glicusio (6 Carboni) a piruvato (3 Carboni). Pu\`o essere suddivisa in 3 parti, che a loro volta racchiudonno 10 reazioni enzimatiche:
\begin{enumerate}
    \item Investimento energetico (1-3);
    \item Rottura della molecola (4-5);
    \item Conservazione dell'energia (6-10).
\end{enumerate}
Vengono formati 4 ATP e consumti 2 ATP, quindi il bilancio netto \`e die 2 ATP. Due molecole di NAD\ap{+}, invece, vengono ridotte a NADH.
\\Gli step della glicolisi sono:
\begin{itemize}
    \item (1) Fosforilazione del glucosio (ATP $\xrightarrow{}$ ADP) per formare glucosio-6-fosfato. L'enzima utilizzato \`e l'esochinasi.
    \item (2) Fosforilazione ( ATP $\xrightarrow{}$ ADP). L'ezima utilizzato \`e l'isomerasi.
    \item (3) Isomerizzazione per formare fruttosio 1,6-bifosfato. L'enzima utilizzato \`e il fosfofruttachinasi.
    \item (4) Il fruttosio 1,6-bifosfato viene tagliato per formare gliceraldeide 3-fosfato (G3P) e diidrossiacetone fosfato (DHAP). Sono delle molecole con 3 Carboni. L'enzima utilizzato \`e l'aldolasi.
    \item (5) Il DHAP viene isomerizzato a G3P, grazie all'enzima triosofosfato isomerasi.
    \item (6) Dopo l'aggiunta di 2 fosfati c'è la formazione di 2 NADH da 2 NAD\ap{+}. Si forma il 1,3-difosfoglicerico. L'enzima impiegato \`e il G3P deidrogenasi.
    \item (7) Formazione di 2 ATP da 2 ADP.
    \item (8-9) Rilascio di 2 molecole di $H_2$O e isomerizzazione con conseguente produzione di fosfoenolpiruvato (PEP).
    \item (10) Viene tolto l'ultimo fosfato dalle 2 PEP per formare 2 ATP e formazione di 2 piruvati. In questo step avviene un passaggio diretto di un fosfato dal PEP a una molecola di ADP. Questo \`e mediato da un enzima, con attraccato Mg\ap{2+}; il complesso \`e quindi un oloenzima.
\end{itemize}
\section{Respirazione cellulare}
Durante il processo della respirazione cellulare \`e prevista la degradazione completa della molecola, in seguito ad una serie di reazioni redox. Le tre fasi sono: 
\begin{enumerate}
    \item sintesi di acetyl-CoA;
    \item ciclo di Krebs; 
    \item una sequenza di reazioni redox detta catena di trasporto elettronico.
\end{enumerate}
\subsection{Sintesi acetilCoA}
L'enzima decarbossilasi rimuove un atomo di carbonio dall'acido piruvato sotto forma di $CO_2$, poi media l'attacco con l'acetato al coenzima-A con un legame ad alta energia. In questo ultimo processo una molecola di NAD\ap{+} \`e ridotta a NADH.
\subsection{Ciclo di Krebs}
Il ciclo di Krebs o ciclo dell'acido citrico \`e composto da 8 reazioni enzimatiche che trasferiscno l'energia contenuta nei legami dell'acetyl-CoA ai coenzimi NAD e FAD, infine riducendoli.Si pu\`o dividere nei seguneti passaggi:
\begin{itemize}
    \item (1) Acetyl-CoA entra nel ciclo unendosi all'acido ossaloacetico per formare acido citrico;
    \item (2-4) due ossidazioni e decarbossilazioni e l'aggiunta di CoA formano il succinyl-CoA
    \item (5) Fosforilazione a livello di substrato
    \item (6-8) Ulteriori ossidazioni regenerano l'acido ossaloacetico
\end{itemize}
Durante lo step 5, una piccola parte dell'energia del processo viene stoccata in 1 molecola di ATP grazie alla fosforilazione, utilizzando il GTP prodotto come intermedio.
\\La maggior parte dell'energia viene conservata sotto forma di elettroni in reazioni redox, nei traportatori di elettroni NADH (step 3, 4, 8) e FADH (step 6). L'energia di questi elettroni servir\`a a valle per produrre ATP.
\\Va ricordato che vengono prodotte anche 3 molecole di $CO_2$. 
\\Il ciclo di Krebs ha anche un ruolo biosintetico, infatti molti dei suoi intermedi possono essere usati come precursori per la costruzione di altre molecole (amminoacidi, anelli porfirinici dei citocromi...). Questo vale anche per la glicolisi. 
\subsection{Catena di trasporto degli elettroni}
Durante questa fase, indirettamente, viene prodotta la maggior parte di ATP. Non c'\`e un vero e proprio passaggio di gruppi fosfato in questa catena, ma essi generano un gradiente che viene sfruttato per fosforilare l'ADP. 
\\La catena di trasporto elettronico consiste in una serire di molecole associate alla membrana che a turno ricevono e cedono elettroni fino ad un accettatore finale di elettroni. L'energia prodotta viene utilizzata per pompare elettroni attraverso la membrana, creando la forza proton-motrice. 
\\Tipi di molecole carrier nella catena di trasporto ellettronico:
\begin{itemize}
    \item Flavoproteine $\xrightarrow{}$ sono delle proteine integrali di membrane che contengono il coenzima flavina, molecola derivata dalla vitamina $B_2$. La flavina mononucleotide (FMN) \`e l'accettatrice iniziale di elettroni. Accettano i 2H\ap{+} e 2 elettroni, ma donano soltanto elettroni. Come tutti gli altri componenti della catena alternano tra stato ridotto e ossidato;
    \item Proteine ferro-zolfo $\xrightarrow{}$ \`e un gruppo di proteine di membrana che contiene ioni metallici (Fe e S) i quali si ossidano e riducono durante il passaggio di elettroni. Come i citocromi trasportano solamente elettroni; i ferri si legano a cisteine. Possono alternare tra stato ridotto e ossidato;
    \item Ubichinone $\xrightarrow{}$ sono dei carrier non proteici derivati dalla vitamina K e altamente idrofobici, vengono chiamati anche Coenzima Q. Accettano i 2H\ap{+} e 2 elettroni;
    \item Citocromi $\xrightarrow{}$ sono delle proteine integrali. Legano un gruppo eme, che \`e costituito da un annello porfirinico e un atomo di ferro. Il ferro altrena tra stato ridotto (Fe\ap{2+}) ed ossidato (Fe\ap{3+}). Il citocromo c serve da intermedio far il bc e l'aa. Subiscono delle ossidazioni e riduzioni mediante la perdita o l'acquisto di un singolo elettrone da parte del ferro. Sono diversificati e possono formare complessi fra loro, come il citocromo $bc_1$.
\end{itemize}
Alla fine gli elettroni vengono accettati dall'ossigeno. La reazione tra elettroni, ossegeno e H\ap{+} forma l'acqua. Proprio per questa specifica funziona l'ossigeno \`e molto importante per un organismo ed \`e proprio questo il motivo per cui respiriamo.
\\Man mano che gli elettroni si spostano lungo la catena, il livello dell'energia si abbassa e vengono pomapati ioni H\ap{+} all'esterno. QUesti derivano o dai trasporttori di elettroni NADH e FADH o dall'idrolisi dell'acqua nei citocromi. Questi protoni passando attraverso l'ATP sintasi, la quale gener ATP. Ogni 2 H\ap{+} viene, approssimativamente, generato un ATP.
\subsection{ATPsintasi}
L'ATPsintasi o ATPasi \`e un grande complesso enzimatico di membrana che serve da catalizzatore della conversione della forza proton-motrice in ATP. Contiene due porzioni principali, una testa con subunit\`a multiple ($F_1$), collocata nella faccia citoplasmatica della memrana, e un canale conduttore di protoni ($F_0$) che attraversa la membrana. Questo complesso, $F_1$/$F_0$ catalizza una reazione da ADP+Pi verso ATP.
\\$F_0$:
\begin{itemize}
    \item Subunit\`a a: \`e il canale attravero il quale passano gli ioni H\ap{+}, che provoca il movimento delle subunit\`a c;
    \item Subunit\`a c: \`e il rotore ed \`e composto da 12-15 subunit\`a singole. \`E la sua torsione che provoca dei movimenti e dei cambiamenti conformazionali che permettono di generare ATP.
\end{itemize}
$F_1$:
\begin{itemize}
    \item Subunit\`a $\varepsilon$ e $\gamma$: connettono il rotore alla parte pi\`u massiccia di $F_1$. La torsione della subunit\`a c genera la rotazione accoppiata dalle due subunit\`a;
    \item Subunit\`a $\alpha$: sono 3 e hanno principalmente una funzione strutturale;
    \item Subunit\`a $\beta$: sono 3, sono alternate a quelle $\alpha$ e hanno il ruolo di sintetizzare. Hanno tutte la stessa funzione, ma la attuano a turno: la prima \`e vuota, la seconda contiene ATP+P e la terza contiene solo ATP. Poi il ciclo si sposta avanti e ritorna come partito. L'ATP viene generato quando la subunit\`a torna alla sua conformazione originale;
    \item Subunit\`a $b_2$ e $\delta$: 
\end{itemize}
L'ATP pu\`o funionare anche al contrario: idrolizzando l'ATP e pompando all'esterno H\ap{+}.
\section{Bilancio globale}
\textbf{Glicolisi}: Glucosio + NAD\ap{+} + 2ATP $\xrightarrow{}$ 2 piruvato + 4ATP + 2NADH
\\Dato che 1 NADH $\xrightarrow{}$ 3 ATP, si ha un totale di 8 ATP prodotti: 2 dalla fosforilazione a livello del substrato e 6 dalla respirazione cellulare del NADH.
\\\textbf{Ciclo di Krebs}: Piruvato + 4 NAD\ap{+} + FAD $\xrightarrow{}$ 4 NADH + 1 FADH + 1 GTP + $3CO_2$
\\Dato che da un FADH si ottengono 2ATP, si ha: 1ATP generato dalla fosforilazione a livello di substrato, e 14 ATP dalla respirazione dei 4 NADH e del FADH. Questi 15 ATP devono essere moltiplicati per due, poich\`e nel circolo entrano 2 molecole di piruvato, ottenendo cos\`i 30 ATP.
\\In totale, da una molecola di glucosio si ottengono circa 38 ATP e ogni giorno un uomo genera una quantit\`a di ATP pari alla sua massa.
\section{Le altarnative cataboliche}
I microrganismi anaerobi non hanno l'ossigeno come accettore finale di elettroni. L'ossigeno viene sostituito con un'altra molecola: $SO_4\ap{2-}$ per ridurlo ad $H_2$S; altri riducono carbonati $CO_3\ap{-}$ a metano $CH_4$; altri utilizzano i nitrati $NO_3\ap{-}$ per produrre $N_2$ o $N_2$O. Quando vengono utilizzati questi trasportatori al posto dell'ossigeno si verifica una perdita di energia. perch\`e hanno $E_0$ meno positivo. Gli aerobi e anaerobi facoltativi sono in grado di utilizzare entrambe le vie.
\\La chemiolitotrofia prevede l'utilizzo di sostanza inorganiche come donatori di elettroni (come FAD o NAD). Esempi sono l'idrogeno solforato ($H_2$S), idrogeno gassoso ($H_2$), ferro ferroso (Fe\ap{2+}), ammoniaca ($NH_3$). I chemiorganotrofi usano come unica fonte il carbonio per produrre energia e per la biosintesi composti organici. I chemilitotrofi utilizzano anidride carbonica per la biosintesi delle loro molecole e composti inorganici per produrre energia. 
\\Nella fototrofia viene utilizzata come fonte di energia la luce; mentre l'ATP viene generato tramite il processo di fosforilazione. I fotoautotrofi assimilano $CO_2$ come fonte di carbonio; mentre i fotoeterotrofi usano come fonte di carbonio composti inorganici. 
\\Esistono due tipi di fotosintesi: ossigenica, che nei cianobatteri produce $CO_2$ e annossigenica.
\\La diversit\`a metabolica nella respirazione e nella fotosintesi ruota intorno alla generazione della forza proton-motrice. \\Nella fermentazione, la fosforilazione avviene solamente al libvello del substrato e quindi dipende dalla forza proton-motrice.
\section{La fermentazione}
\`E una via metabolica alternativa in caso di mancanza di un accettatore finale di elettroni nel processo di respirazione cellulare. Se manca un accettatore finale di elettroni tutta la via respiratoria si blocca. L'ATP necessario potrebbe essere sintetizzato dalla glicolisi e dal ciclo di Krebs, ma questo non pu\`o succedere perch\`e entrambi i processi richiedonon di NAD\ap{+}. 
\\La fermentazione avviene grazie all'aggiunta di 2H a un piruvato per formare acido lattico; oppure grazie alla decarbossilazione del piruvato e alla successiva aggiunta di 2H per produrre etanolo. La fermentazione non produce direttamente ATP, ma fa si che la sua produzione possa avvenire durante il ciclo di Krebs. Anche se l'energia stocatta \`e in quantit\`a minore rispetto a quella della respirazione, essa consente di produrre ATP senza un accettatore di elettroni. 
\subsection{Fermentazione lattica}
Nel processo della fermentazione lattica i due atomi di idrogeno vengono trasferiti sul carbonio in posizione 2 dell'acido piruvico, producendo l'acido lattico.
\\\begin{center}CH$_3$COCOOH\ap{-} + NADH + H\ap{+} $\xrightarrow{}$ CH$_3$HCOHCOOH + NAD\ap{+}\end{center} 
\`E un processo che viene attuato da alcuni batteri, come i lattobacilli, e dalle cellule del corpo umano in condizioni di anaerobiosi, come i muscoli. 
\subsection{Fermentazione alcolica}
1\ap{a} reazione:
\begin{center}
    C$_3$H$_4$O$_3$ $\xrightarrow{}$ C$_2$H$_4$O + CO$_2$
    \\Acido piruvico $\xrightarrow{}$ Acetaldeide + CO$_2$
\end{center}
2\ap{a} reazione:
\begin{center}
    C$_2$H$_4$O + (NADH + H\ap{+}) $\xrightarrow{+}$ + C$_2$H$_6$O + NAD\ap{+}
    \\Acetaldeide + (NADH + H\ap{+}) $\xrightarrow{}$ +C$_2$H$_6$O + NAD\ap{+} 
\end{center}
Dalla fermentazione del glucosio si possono avere vari prodotti. a glicosi produce piruvato, che pu\`o essere convertito ad acido lattico attraverso la fermentazione lattica o entanolo attraverso la fermentazione alcolica. La fermentazione acido-mista produce una miscela di etanolo, acido lattico, succinico, formico e acetico.
\subsection{Prodotti alimentari o industriali derivati da processi di fermentazione}
\textbf{Pane} $\xrightarrow{}$ Durante la panificazione il lievito fermenta gli oligosaccaridi che si staccano dall'amido durante la fase di impasto e di riposo della massa in lavorazione. I prodotti della fermentazione alcolica (alcol etilico ed anidride carbonica) passano in fase gassosa formando le caratteristiche bolle durante la lievitazione e la cottura.
\\\textbf{Vino} $\xrightarrow{}$ Il vino viene prodotto a partire da soluzioni zuccherine ottenute dallo schiacciamento del grappolo d'uva asciate a fermentare con i lieviti del genere Saccharomyces presenti sulla buccia dell'acino o provenienti da colture selezionate. A seconda delle condizioni di fermentazione, si differenziano le qualit\`a organolettiche (colore, sapori, aromi...)del vino.
\\\textbf{Birra} $\xrightarrow{}$ La birra si ottiene per l'azione di lieviti su un mosto contenente malto di orzo e quantit\`a variabili di altri cereali. La lavorazione \`e tale da consevare nel prodotto anche l'anidride carbonica.
\\\textbf{Yogurt} $\xrightarrow{}$ Lo yogurt \`e il risultato della fermentazione lattica operata da ceppi selezionati di lattobacilli sul latte. L'abbassamento del pH dovuto all'accumulo dell'acido lattico determina la denaturazione della caseina che coaugula conferendo al prodotto la caratteristica consistenza.
\section{Altre vie cataboliche}
Lipidi e proteine contengono una grande quantit\`a di energia nei loro legami. Perch\`e questa energia venga utilizzata dalla cellula, essi devono essere scomposti nei loro monomeri e entrare come substrati nella glicolisi e nel ciclo di Krebs.
\\I \textbf{lipidi} pi\`u utilizzati per la produzione di ATP sono i grassi (composti da glicerolo e code di acidi grassi). Alcuni enzimi chiamati lipasi idrolizzano i lipidi producendo glicerolo e tre catene di acidi grassi. Il glicerolo viene convertito in DHAP che integra la via metabolica della glicolisi, mentre gli acidi grassi sono degradati in un processo chimato $\beta$-ossidazione. 
Durante questo processo degli enzimi tagliano 2 carboni idrogenati che formano le code di acidi grassi, e li uniscono ad una molecola di coenzima. Vengono prodotte cos\`i delle molecole di acetyl-CoA, che pu\`o entrare all'interno del ciclo di Krebs. Il proccesso prosegue finch\`e tutti gli acidi grassi non vengono convertiti in acetyl-CoA. Vengono, quindi, generate delle grandi quantit\`a di trasportatori di elettroni NADH e FADH$_2$, usate per la catena di trasporto degli elettroni.
\\Altri microbi catabolizzano \textbf{proteine} come un importante fonte di energia. La maggior parte delle cellule le catabolizza solo quando non ci sono a disposizioni fonti di carbonio come il glucosio. Dato che le proteine sono troppo grosse per superare la membrana plasmatica, procarioti iniziano a catabolizzarli all'esterno. Gli enzimi protesi degradano le proteine in amminoacidi idrolizzando i legami peptidici. Vengono quindi portati all'interno della cellula e subiscono modificazioni chimiche (deaminazione). La molecola risultante pu\`o entrare nel ciclo di Krebs.
\section{La fotosintesi}
Gli organismi fotosintetici catturano l'energia luminosa e l'utilizzano per la sintesi di carboidrati a partire da CO$_2$ and H$_2$O. I cianobatteri sono stati i primi organismi fotosintetici. Ora anche molte alghe, batteri verdi solfurei e non, piante e alcuni protozoi fanno parte di questo gruppo. Essi riescono a catturare l'energia della luce solare grazie a delle piccole molecole, la pi\`u importante delle quali \`e la clorofilla. La clorofilla \`e forata da una coda idrocarburica idrofobica attaccata a un centro che assorbe uce composto anche da uno ion Mg\ap{2+}. La clorofilla assomiglia ai citocromi, ma al posto del magnesio contengono il ferro al centro dell'anello. 
\\Esistono due tipi di clorofille: 
\begin{itemize}
    \item clorofille che si trovano nelle piante, nelle alghe e nei cianobatteri; 
    \item batterioclorofille che si trovano nei batteri verdi e porpora, e negli eliobatteri.
\end{itemize}
La loro differenza principale \`e la diversa lunghezza d'onda alla quale assorbono. Questa diversit\`a ha determinato diversi habitat in cui gli organismi si sono insediati. 
\\La fotointesi avviene al livello della membrana citoplasmatica dove ci sono molte clorofille a formare i tilacoidi. Le code idrofobiche sono immerse nella membrana, mentre il sito attivo che contiene Mg\ap{2+} \`e esterno ad essa. 
\\I fotosistemi sono formate dall'insieme di clorofille e proteine nella membrana; i tilacoidi dei fotosistemi dei procarioti sono invaginazioni della membrana citoplasmatica e questo permette un aumento di superficie.
\\Esistono due tipi di fotosistemi PSI e PSII. Questi assorbono la luce solare e stoccano l'energia in molecole di ATP e NADPH grazie a reazioni redox. Queste reazioni sono dette dipendenti dalla luce. Tuttavia, ci sono anche delle reazioni non dipendenti dalla luce, nelle quali il glucosio \`e sintetizzato a partire da CO$_2$ e H$_2$O.
\subsection{Reazioni dipendenti dalla luce}
\textbf{Fosforilazione ciclica}
\\\textbf{Fosforilazione non ciclica}
\subsection{Reazioni non dipendenti dalla luce}
